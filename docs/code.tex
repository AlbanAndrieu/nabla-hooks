\documentclass[12pt,a4paper]{article}

\usepackage[utf8]{inputenc}
\usepackage[T1]{fontenc}
\usepackage[french]{babel}
%\usepackage[scale=0.8]{cascadia-code}
\usepackage{minted}
\setminted{
  linenos, numbersep=2mm, stepnumber=2, stepnumberfromfirst,
  frame=lines, framerule=2pt, rulecolor=lightgray, bgcolor=myGray
}
\usepackage{xcolor}
\usepackage{fontawesome5}

\definecolor{myGray}{RGB}{237,235,235}

\begin{document}

\begin{itemize}
  \item Dossier : \faIcon[regular]{folder-open} ou \faIcon[solid]{folder-open}
  \item GitHub : \faIcon[regular]{github} ou \faGithub
  \item YouTub : \faYoutube~et en couleur {\color{red} \faYoutube}
\end{itemize}

\bigskip

Voici un example de code~:
\begin{listing}[H]
\begin{minted}[
  linenos, numbersep=2mm, stepnumber=2, stepnumberfromfirst,
  frame=lines, framerule=2pt, rulecolor=lightgray, bgcolor=myGray
]
{python}
def hello() -> None:
  print('Hello World!')

if __name__ == '__main_':
  hello()
\end{minted}
\end{listing}

\bigskip

\inputminted[]{python}{conf.py}
\end{document}
